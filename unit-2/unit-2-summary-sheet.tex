\documentclass{article}
\usepackage[landscape]{geometry}
\usepackage{url}
\usepackage{multicol}
\usepackage{amsmath}
\usepackage{esint}
\usepackage{amsfonts}
\usepackage{tikz}
\usetikzlibrary{decorations.pathmorphing}
\usepackage{amsmath,amssymb}
\usepackage{graphicx}
\usepackage{float}

\usepackage{colortbl}
\usepackage{xcolor}
\usepackage{mathtools}
\usepackage{amsmath,amssymb}
\usepackage{enumitem}
\makeatletter

\newcommand*\bigcdot{\mathpalette\bigcdot@{.5}}
\newcommand*\bigcdot@[2]{\mathbin{\vcenter{\hbox{\scalebox{#2}{$\m@th#1\bullet$}}}}}
\makeatother

\title{Unit 1 Summary Sheet}
\usepackage[T1]{fontenc}
\usepackage[utf8]{inputenc}
\usepackage[english]{babel}

\advance\topmargin-.8in
\advance\textheight3in
\advance\textwidth3in
\advance\oddsidemargin-1.5in
\advance\evensidemargin-1.5in
\parindent0pt
\parskip2pt
\newcommand{\hr}{\centerline{\rule{3.5in}{1pt}}}
%\colorbox[HTML]{e4e4e4}{\makebox[\textwidth-2\fboxsep][l]{texto}
\begin{document}

\begin{center}{\huge{\textbf{Unit 2 Summary Sheet}}}\\
\end{center}
\begin{multicols*}{3}

\tikzstyle{mybox} = [draw=black, fill=white, very thick,
    rectangle, rounded corners, inner sep=10pt, inner ysep=10pt]
\tikzstyle{fancytitle} =[fill=black, text=white, font=\bfseries]

%------------ Types of Reactions ---------------
\begin{tikzpicture}
\node [mybox] (box){%
    \begin{minipage}{0.3\textwidth}
    \begin{enumerate}
        \item \textbf{Synthesis reaction}: occurs when two or more elements combine to form a compound. \\
		e.g. $2Na(s)+Cl_{2}(g)\rightarrow 2NaCl(s)$ \\
		$A+B\rightarrow AB$
        \item \textbf{Decomposition reaction}: occurs when a compound breaks down into two or more simpler substances. \\
		e.g. $2AgCl(s)\rightarrow 2Ag(s)+Cl_{2}(g)$ \\
		$AB\rightarrow A+B$
        \item \textbf{Single Displacement reaction}: occurs when an element replaces another element in a compound. \\
		e.g. $Zn(s)+2HCl(aq)\rightarrow ZnCl_{2}(aq)+H_{2}(g)$ \\
		$AB+C\rightarrow AC+B$
        \item \textbf{Double Displacement reaction}: occurs when the cations and anions displace each other to form two new products. \\
		e.g. $AgNO_{3}(aq)+NaCl(s)\rightarrow AgCl(s)+NaNO_{3}(aq)$ \\
		$AB+CD\rightarrow AD+CB$
	\item \textbf{Complete combustion reaction}: occurs when a hydrocarbon reacts with a sufficient amount of oxygen gas to produce carbon dioxide and water (blue flame). \\
		$C_{x}H_{y}(g)+O_{2}(g)\rightarrow CO_{2}(g)+H_{2}O(g)$
	\item \textbf{Incomplete combustion reaction}: occurs when a hydrocarbon reacts with an insufficient amount of oxygen gas to produce carbon dioxide, water and carbon monoxide (orange flame). \\
		$C_{x}H_{y}(g)+O_{2}(g)\rightarrow CO(g)+CO_{2}(g)+H_{2}O(g)$
    \end{enumerate}
    \end{minipage}
};
%------------ Types of Reactions Header ---------------------
\node[fancytitle, right=10pt] at (box.north west) {Types of Reactions};
\end{tikzpicture}

%------------ Balancing Chemical Equations ---------------
\begin{tikzpicture}
\node [mybox] (box){%
    \begin{minipage}{0.3\textwidth}
    \textbf{Law of Conservation of Mass}: mass in an isolated system is neither created nor destroyed by chemical reactions or physical transformations. \\
    \textbf{Skeleton Equation}: chemical equation in which the number of atoms is not equal on both sides. \\
    \textbf{Balanced Equation}: number of atoms in the \textit{reactants} must equal the number of atoms in the \textit{products}. \\
    e.g. $N_{2}(g)+3H_{2}(g)\rightarrow 2NH_{3}(g)$
    \end{minipage}
};
%------------ Balancing Chemical Equations Header ---------------------
\node[fancytitle, right=10pt] at (box.north west) {Balancing Chemical Equations};
\end{tikzpicture}

%------------ Activity Series ---------------
\begin{tikzpicture}
\node [mybox] (box){%
    \begin{minipage}{0.3\textwidth}
    \textbf{The Activity Series}: used for predicting single displacement reactions.
    \begin{itemize}
        \item The element on the reactant side must be more ``active'' than the one it could replace in order for a single displacement reaction to occur. \\
	e.g. $Al(s)+CuCl_{2}(aq)\rightarrow Cu(s)+AlCl_{3}(aq)$
        \item If the element is not more ``active'' than the one it could replace, then the single displacement reaction will not occur. \\
	e.g. $Ni(s) + NaCl(aq)\rightarrow$ NR (no reaction)
	\item The reactivity of $H_{2}O$ depends on the metal and water temperature.
	\item Near the top, the metals are more reactive because their valence electrons are easier to be removed.
	\item There is an activity series just for non-metals, this is much shorter.
    \end{itemize}
    \end{minipage}
};
%------------ Activity Series Header ---------------------
\node[fancytitle, right=10pt] at (box.north west) {Activity Series};
\end{tikzpicture}

%------------ Solubility Rules ---------------
\begin{tikzpicture}
\node [mybox] (box){%
    \begin{minipage}{0.3\textwidth}
    \textbf{Solubility Rules}: used for predicting what products will be soluble or insoluble. \\ \\
    A \textbf{double displacement reaction} has occurred when there is:
    \begin{itemize}
        \item a formation of a precipitate (s),
	\item a formation of a gas (g),
	\item a formation of water (neutralization reaction).
    \end{itemize}
    \textbf{Molecular Equation}: \\
     $KBr(aq)+AgNO_{3}(aq)\rightarrow KNO_{3}(aq)+AgBr(s)$ \\ \\
    \textbf{Complete Ionic Equation}: all reactants and products separated into ions, contains spectator ions. \\
    ${K}^{+}+{Br}^{-}+{Ag}^{+}+{NO_{3}}^{-}\rightarrow {K}^{+}+{NO_{3}}^{-}+{Ag}^{+}+{Br}^{-}$ \\ \\
    \textbf{Net Ionic Equation}: eliminates the ions not directly involved in making the reaction happen. \\
    ${Br}^{-}+{Ag}^{+}\rightarrow AgBr(s)$ \\
    \end{minipage}
};
%------------ Solubility Rules Header ---------------------
\node[fancytitle, right=10pt] at (box.north west) {Solubility Rules};
\end{tikzpicture}

%------------ The Mole ---------------
\begin{tikzpicture}
\node [mybox] (box){%
    \begin{minipage}{0.3\textwidth}
    \textbf{Avogadro's constant ($N_{A}$)}: $6.022\times{10}^{23}$, this is how many particles are in one mole. The same number of atoms in 12 grams of carbon-12. \\
    \textbf{Mole}: used to count atoms, ions, molecules, or formula units in groups of $6.022\times{10}^{23}$. A mole of pure carbon-12 is exactly 12 grams. \\
    \textbf{Molecule}: basic unit of a covalent substance. \\
    \textbf{Formula Unit}: basic unit of an ionic substance.
    \begin{itemize}
        \item $n$ = number of moles
	\item $N_{A}$ = Avogadro's constant
	\item $N$ = number of particles
	\item $\displaystyle{n=\frac{N}{N_{A}}}$
	\item $N=n\times N_{A}$
    \end{itemize}
    e.g. How many gold atoms in $4.70\times{10}^{-4}$ mol of gold? \\
    $N=4.70\times{10}^{-4}\,mol\times6.022\times{10}^{23}\,atoms/mol$ \\
    $N\approx2.83\times{10}^{20}\,Au\,Atoms$
    \end{minipage}
};
%------------ The Mole Header ---------------------
\node[fancytitle, right=10pt] at (box.north west) {The Mole};
\end{tikzpicture}

%------------ Molar Mass ---------------
\begin{tikzpicture}
\node [mybox] (box){%
    \begin{minipage}{0.3\textwidth}
    \textbf{Molar mass (M)}: the mass in grams of one mole of a substance. It is numerically the same as the \textbf{atomic mass unit (amu)} of an element, except the units are in grams per mole (g/mol). \\
    e.g. Atomic mass of Calcium = 40.078 amu \\
    e.g. Molar mass of Calcium = 40.078 g/mol \\
    \textbf{Representation}: A mole ($6.022\times{10}^{23}$ Ca atoms) of calcium atoms weigh 40.078 grams. \\ \\
    e.g. What is the molar mass of $H_{2}O$? \\
    $M_{H_{2}O}=2\times1.0079\,g/mol+15.9994\,g/mol$ \\
    $M_{H_{2}O}\approx18.02\,g/mol$
    \begin{itemize}
        \item $n$ = number of moles (mol)
        \item $M$ = molar mass (g/mol)
        \item $m$ = mass (g)
	\item $\displaystyle{M=\frac{m(g)}{n(mol)}}$, $\displaystyle{n=\frac{m}{M}}$, $m=n\times M$
    \end{itemize}
    \end{minipage}
};
%------------ Molar Mass Header ---------------------
\node[fancytitle, right=10pt] at (box.north west) {Molar Mass};
\end{tikzpicture}

%------------ Percent Composition ---------------
\begin{tikzpicture}
\node [mybox] (box){%
    \begin{minipage}{0.3\textwidth}
    \textbf{Law of Definite Proportions}: the proportions of each element in a chemical compound, regardless of quantity, are the same. \\
    \textbf{Percent Composition}: the percent by mass of each element in a compound.
    $$\%\,mass\,of\,element=\frac{mass\,of\,element}{total\,mass\,of\,compound}\times100\%$$
    e.g. What is the percent composition of $CO_{2}$? \\
    $M_{CO_{2}}=44.01\,g/mol$ \\
    $\displaystyle{\,\%C=\frac{12.011g/mol}{44.01\,g/mol}\times100\%}$ \\
    $\%C=27.3\,\%$ \\ \\
    $\displaystyle{\,\%O=\frac{32.00g/mol}{44.01\,g/mol}\times100\%}$ \\
    $\%O=72.7\,\%$
    \end{minipage}
};
%------------ Percent Composition Header ---------------------
\node[fancytitle, right=10pt] at (box.north west) {Percent Composition};
\end{tikzpicture}

%------------ Empirical vs. Molecular Formula ---------------
\begin{tikzpicture}
\node [mybox] (box){%
    \begin{minipage}{0.3\textwidth}
    \textbf{Empirical Formula}: shows the lowest whole number ratio of elements in a compound (simplest formula of a compound). Some compounds can have the same empirical formula. \\
    \textbf{Molecular Formula}: shows the exact number of atoms of each element (actual formula of a compound). Each molecular formula is unique. \\ \\
    Molecular Formula: $C_{6}H_{12}O_{6}$ \\
    Empirical Formula: $CH_{2}O$ \\ \\
    \textbf{Determining the Empirical Formula}:
    \begin{enumerate}
        \item Assume 100.0 g of the compound, change percentages of each of the elements to grams.
	\item Convert the grams of each elements to moles.
	\item Divide by the smallest number of moles.
	\item Multiply by a constant to get a whole number.
    \end{enumerate}
    \textbf{Determining the Molecular Formula}:
    $$ratio=\frac{mass\,of\,molecular\,formula}{mass\,of\,empirical\,formula}$$
    Multiply each element by this ratio in order to determine the molecular formula.
    \end{minipage}
};
%------------ Empirical vs. Molecular Formula Header ---------------------
\node[fancytitle, right=10pt] at (box.north west) {Empirical vs. Molecular Formula};
\end{tikzpicture}

%------------ Molecular Formula of a Hydrate ---------------
\begin{tikzpicture}
\node [mybox] (box){%
    \begin{minipage}{0.3\textwidth}
    \textbf{Hydrate}: ionic compound that contains water molecules in its structure. \\
    \textbf{Anhydrate}: the substance that remains after the water has been removed from the hydrate. \\
    e.g. What is the formula of $MgSO_{4}\cdot XH_{2}O$? \\ \\
    \small{
    \begin{tabular}{lp{3cm} l}
	& $MgSO_{4}$ Hydrate  & 13.52 g \\
        - & Anhydrate  & 6.60 g\\ \hline
	& Water & 6.92 g \\
    \end{tabular}} \\ \\
    $\displaystyle{n_{MgSO_{4}}=\frac{6.60\,g}{120.37\,g/mol}}$ \\
    $n_{MgSO_{4}}=0.054831\,mol$ \\ \\
    $\displaystyle{n_{H_{2}O}=\frac{6.92\,g}{18.02\,g/mol}}$ \\
    $n_{H_{2}O}=0.38402\,mol$ \\ \\
    $\therefore$ dividing $n_{H_{2}O}$ by $n_{MgSO_{4}}$ to determine the coefficient of $H_{2}O$, the hydrate is $MgSO_{4}\cdot7H_{2}O$.
    \end{minipage}
};
%------------ Molecular Formula of a Hydrate Header ---------------------
\node[fancytitle, right=10pt] at (box.north west) {Molecular Formula of a Hydrate};
\end{tikzpicture}

%------------ Stoichiometry ---------------
\begin{tikzpicture}
\node [mybox] (box){%
    \begin{minipage}{0.3\textwidth}
    \textbf{Mole-to-Mole Calculations}: mol A $\rightarrow$ mol B
    \begin{enumerate}
        \item Write the balanced chemical equation.
	\item Write the mole ratios for the given substance(s).
	\item Use the mole ratio to determine what amount of desired substance is produced or needed.
    \end{enumerate}
    \textbf{Mass-to-Mass Calculations}: grams A $\rightarrow$ grams B
    \begin{enumerate}
        \item Write the balanced chemical equation.
	\item Use molar mass to convert mass to moles.
	\item Use mole ratio to find the amount in moles.
	\item Convert moles of desired substance to mass.
    \end{enumerate}
    \textbf{Mass-to-Particle Calculations}: grams A $\rightarrow$ mol B
    \begin{enumerate}
        \item Write the balanced chemical equation.
	\item Use molar mass to convert the given mass to moles.
	\item Apply the mole ratio to determine what amount of desired substance is produced or needed.
	\item Multiply by avogradro's constant to determine the number of particles.
    \end{enumerate}
    \end{minipage}
};
%------------ Stoichiometry Header ---------------------
\node[fancytitle, right=10pt] at (box.north west) {Stoichiometry};
\end{tikzpicture}

%------------ Limiting Reactant ---------------
\begin{tikzpicture}
\node [mybox] (box){%
    \begin{minipage}{0.3\textwidth}
    \textbf{Limiting Reagent}: the first reactant to be used up in a reaction; determines when the reaction stops. \\
    \textbf{Excess Reagent}: the reactant(s) that are not used up when the reaction is finished; it is left over. \\
    \textbf{Stoichiometry involving Limiting Reactant}: identifying which reactant will run out first.
    \begin{enumerate}
	\item Write the balanced chemical equation.
	\item Convert each reactant to moles.
	\item Determine the amount of desired product that can be produced using the mole ratio.
	\item The limiting reactant is the one that produces the least amount of product.
	\item Using the amount of product based on the limiting reactant, convert the moles of the desired substance into mass.
    \end{enumerate}
    \end{minipage}
};
%------------ Limiting Reactant Header ---------------------
\node[fancytitle, right=10pt] at (box.north west) {Limiting Reactant};
\end{tikzpicture}

%------------ Percent Yield ---------------
\begin{tikzpicture}
\node [mybox] (box){%
    \begin{minipage}{0.3\textwidth}
    \textbf{Theoretical Yield}: calculated amount of product based on the stoichiometry of the reaction. \\
    \textbf{Actual Yield}: amount of product collected during the experiment. \\
    Usually, the theoretical yield $>$ actual yield. \\
    \textbf{Percent Yield}: the percent ratio of the actual yield to the theoretical yield.
    $$percent\,yield=\frac{actual\,yield}{theoretical\,yield}\times100\%$$
    \textbf{Factors affecting percentage yield}:
    \begin{itemize}
        \item Reversible reactions.
	\item Not enough time to complete the reaction.
	\item Reactants contain impurities.
    \end{itemize}
    \textit{Too high of a temperature does not affect percent yield.}
    \end{minipage}
};
%------------ Percent Yield Header ---------------------
\node[fancytitle, right=10pt] at (box.north west) {Percent Yield};
\end{tikzpicture}

%------------ Percent Purity ---------------
\begin{tikzpicture}
\node [mybox] (box){%
    \begin{minipage}{0.3\textwidth}
    \textbf{Percent Purity}: the percent ratio of the pure product to the impure product obtained by mass.
    $$\%\,purity=\frac{mass\,of\,pure\,product}{mass\,of\,impure\,product\,obtained}\times100\%$$
    \end{minipage}
};
%------------ Percent Purity Header ---------------------
\node[fancytitle, right=10pt] at (box.north west) {Percent Purity};
\end{tikzpicture}

\end{multicols*}
\end{document}
